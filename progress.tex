\documentstyle[12pt,psfig]{article}
\textwidth=6in
\textheight=8in
\topmargin=0.5in
\oddsidemargin=0.25in
\evensidemargin=0.25in
\headheight=0in
\footheight=0in
\headsep=0in

\begin{document}
\centerline{\Large \bf Progress Report}
\bigskip
\bigskip

\noindent {\bf Group Number:} G26
\bigskip

\noindent {\bf Project Name:} {\it Distributed Automotive Sensor/Actuator Network}
\bigskip

\noindent {\bf Reporting Period:} 01/09/2009 to 22/10/2009
\bigskip

\noindent {\bf Date submitted:} \today

\begin{tabbing}
 \hspace{50pt}\=\kill
 {\bf To:} \> Dr. W. Kinsner, University of Manitoba \\ 
 {\bf From:} \> John Hughes, Mike Jean \\
 {\bf CC:} \> C. Shafai, etc.
\end{tabbing}

\hrule
\bigskip

\paragraph{Introduction:}

The major goals of the project include design and implementation of the CAN tester, hardware design for the four major modules, manufacturing and populating of the four module printed circuit boards, design and writing of control software, testing of individual modules, and completing the neccessary reports for the thesis course. 

The project is currently on schedule. Module hardware design is nearly completed and several of the test suites have been  written. The CAN network tester has been designed and preliminary software has been written.

\paragraph{Summary of Work Completed:}

\begin{enumerate} 

Much further reasearch has been done on the clutch control system. Two seperate tests have been performed on the system: a bench test of a PWM solenoid valve to prove the idea that variation in PWM percentage with a constant supply pressure would modulate the steady-state force on the pison, and therefore the clutch position. 

MFJ has written a series of software test suites to verify the correct operaiton of each module. He has also written burn-in tests to verify that the PCB design is correct and all components have been populated correctly. 

MFJ has designed the CAN testing module, and will be using an Olimex AT90CAN128 development board as the platform for the embedded component of the module. The client-embedded protocol has been designed. The interface between the client and embedded system has been designed and test applications have been written. A small CAN receiver module that blinks an LED when receiving a message has been designed to test the module.

\paragraph{Future Work:}

The following tasks will be completed for the next reporting period:

\begin{enumerate}

\item The client and embedded software for the CAN testing unit will be written. (MFJ)

\item This requires an AT90CAN128 development board and JTAG programmer from Olimex. A past SAE sponsor has been contacted to provide both the development board and the programmer. Expected reply is October 26th, 2009. Parts will be ordered on the same day and received by October 30th, 2009. Total expendature is \$111.91, although it is entirely sponsored.

\item The hardware design for the wireless telemetry, driver interface, and brake modules will be completed. (JCH)

\item All software test suites to verify hardware functionality will be written. (MFJ) 

\item Components will be ordered from DigiKey. Total cost for components is estimated at \$446.82, but the cost is entirely sponsored. Components will be ordered before October 30th, 2009. Delivery is expected by November 2nd, 2009.

\item Printed circuit board layouts will be sent to the manufacturer for production. (JCH) Total cost for manufacture of circuit boards is \$400.00, but the cost is covered by sponsorship. Schematics will be sent by November 2nd, 2009. Expected delivery date is November 7th, 2009.

\item Printed circuit boards will be populated with components. (JCH, MFJ)
\item Printed circuit boards will be tested with the software test suites to verify functionality. (JCH, MFJ)

\end{enumerate}

\end{document}

