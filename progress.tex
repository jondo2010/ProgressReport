\documentstyle[12pt,psfig]{article}
\textwidth=6in
\textheight=8in
\topmargin=0.5in
\oddsidemargin=0.25in
\evensidemargin=0.25in
\headheight=0in
\footheight=0in
\headsep=0in

\begin{document}
\centerline{\Large \bf Progress Report}
\bigskip
\bigskip

\noindent {\bf Group Number:} G26
\bigskip

\noindent {\bf Project Name:} {\it Distributed Automotive Sensor/Actuator Network}
\bigskip

\noindent {\bf Reporting Period:} 01/09/2009 to 22/10/2009
\bigskip

\noindent {\bf Date submitted:} \today

\begin{tabbing}
 \hspace{50pt}\=\kill
 {\bf To:} \> Dr. W. Kinsner, University of Manitoba \\ 
 {\bf From:} \> John Hughes, Mike Jean \\
 {\bf CC:} \> C. Shafai, etc.
\end{tabbing}

\hrule
\bigskip

\paragraph{Introduction:}
Introductory paragraph (up to 150 words) which gives the overall objectives of the project and whether or not things are going as planned. The reader should know right away if there are problems or if everything is on schedule.

\paragraph{Summary of Work Completed:}

\subparagraph{Clutch control system} Much further reasearch has been done on the clutch control system. \\
Two seperate tests have been performed on the system:
\begin{enumerate}
 \item a bench test of a PWM solenoid valve to prove the idea that variation in PWM percentage with a constant supply pressure would modulate the steady-state force on the pison, and therefore the clutch position. The central compressed air supply was used (120psi), and a micro was set up on a breadboard to generate a PWM signal fed into an amplifier stage for the valve. Varying the PWM percentage did in fact modulate the position of the clutch, however the time constant to reach steady-state was very long. This is related to the flow rate through the valve. We were able to determine that the particular valve, although particularly suited to PWM control signals, had too small of a flow rate for our requirements.
 \item 
\end{enumerate}

Summarize the work completed so far. Specify the work performed by each member as much as
possible. (You are expected to spend 6 hours per week on your project).
Refer to the task names in the Proposal and GNATT chart if they are good indicators for what was done;
otherwise give a brief description of the tasks which were actually carried.
Mention the dates of any design review meetings which were held and describe any decisions made and
reasons for making those decisions (demonstrate how you applied engineering knowledge in the
decision making process).
Describe the problems that were encountered (if any). State if you have made any modifications to the
tasks you have completed as compared to what was stated in the Proposal.


\paragraph{Future Work:}

Briefly describe the future work for the next reporting period. If there are major changes with respect to
the Proposal then you will need to go into some depth here (if everything is still as in the proposal then
be brief). Specify the work to be performed by each member as much as possible.
If a revised GANTT chart is required then include this as an attachment to the progress report.
If you have to order parts state whether or not all parts have been received and what the total
expenditure was. If you are still waiting for parts give the expected delivery dates.


\end{document}

