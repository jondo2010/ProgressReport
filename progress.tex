\documentstyle[12pt,psfig]{article}
\textwidth=6in
\textheight=8in
\topmargin=0.5in
\oddsidemargin=0.25in
\evensidemargin=0.25in
\headheight=0in
\footheight=0in
\headsep=0in

\begin{document}
\centerline{\Large \bf Progress Report}
\bigskip
\bigskip

\noindent {\bf Group Number:} G26
\bigskip

\noindent {\bf Project Name:} {\it Distributed Automotive Sensor/Actuator Network}
\bigskip

\noindent {\bf Reporting Period:} 01/09/2009 to 22/10/2009
\bigskip

\noindent {\bf Date submitted:} \today

\begin{tabbing}
 \hspace{50pt}\=\kill
 {\bf To:} \> Dr. W. Kinsner, University of Manitoba \\ 
 {\bf From:} \> John Hughes, Mike Jean \\
 {\bf CC:} \> C. Shafai, etc.
\end{tabbing}

\hrule
\bigskip

\paragraph{Introduction:}

The major goals of the project include design and implementation of the CAN tester, hardware design for the four major modules, manufacturing and populating of the four module printed circuit boards, design and writing of control software, testing of individual modules, and completing the neccessary reports for the thesis course. 

The project is currently on schedule. Module hardware design is nearly completed and the test suites have been designed and written. The CAN network tester has been designed and preliminary software has been written.

\paragraph{Summary of Work Completed:}

\subparagraph{Clutch control system} Much further reasearch has been done on the clutch control system. \\
Two seperate tests have been performed on the system:
\begin{enumerate}
 \item a bench test of a PWM solenoid valve to prove the idea that variation in PWM percentage with a constant supply pressure would modulate the steady-state force on the pison, and therefore the clutch position. The central compressed air supply was used (120psi), and a micro was set up on a breadboard to generate a PWM signal fed into an amplifier stage for the valve. Varying the PWM percentage did in fact modulate the position of the clutch, however the time constant to reach steady-state was very long. This is related to the flow rate through the valve. We were able to determine that the particular valve, although particularly suited to PWM control signals, had too small of a flow rate for our requirements.
 \item 
\end{enumerate}

Summarize the work completed so far. Specify the work performed by each member as much as
possible. (You are expected to spend 6 hours per week on your project).
Refer to the task names in the Proposal and GNATT chart if they are good indicators for what was done;
otherwise give a brief description of the tasks which were actually carried.
Mention the dates of any design review meetings which were held and describe any decisions made and
reasons for making those decisions (demonstrate how you applied engineering knowledge in the
decision making process).
Describe the problems that were encountered (if any). State if you have made any modifications to the
tasks you have completed as compared to what was stated in the Proposal.


\paragraph{Future Work:}

The following tasks will be completed for the next reporting period:

\begin{enumerate}

\item The client and embedded software for the CAN testing unit will be written. (MFJ)

\item This requires an AT90CAN128 development board and JTAG programmer from Olimex. A past SAE sponsor has been contacted to provide both the development board and the programmer. Expected reply is October 26th, 2009. Parts will be ordered on the same day and received by October 30th, 2009. Total expendature is \$111.91, although it is entirely sponsored.

\item The hardware design for the wireless telemetry, driver interface, and brake modules will be completed. (JCH)

\item All software test suites to verify hardware functionality will be written. (MFJ) 

\item Components will be ordered from DigiKey. Total cost for components is estimated at \$446.82, but the cost is entirely sponsored. Components will be ordered before October 30th, 2009. Delivery is expected by November 2nd, 2009.

\item Printed circuit board layouts will be sent to the manufacturer for production. (JCH) Total cost for manufacture of circuit boards is \$400.00, but the cost is covered by sponsorship. Schematics will be sent by November 2nd, 2009. Expected delivery date is November 7th, 2009.

\item Printed circuit boards will be populated with components. (JCH, MFJ)
\item Printed circuit boards will be tested with the software test suites to verify functionality. (JCH, MFJ)

\end{enumerate}

\end{document}

